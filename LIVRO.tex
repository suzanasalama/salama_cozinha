\documentclass[showtrims,		% Mostra as marcas de corte em cruz
			   %trimframe,		% Mostra as marcas de corte em linha, para conferência
			   11pt				% 8pt, 9pt, 10pt, 11pt, 12pt, 14pt, 17pt, 20pt 
			   ]{memoir}
\usepackage[brazilian,
			% english,
			% italian,
			% ngerman,
			% french,
			% russian,
			% polutonikogreek
			]{babel}
\usepackage{anyfontsize}			    % para tamanhos de fontes maiores que \Huge 
\usepackage{relsize}					% para aumentar ou diminuir fonte por pontos. Ex. \smaller[1]
\usepackage{fontspec}					% para rodar fontes do sistema
\usepackage[switch]{lineno} 			% para numerar linhas
\usepackage{lipsum}						% para colocar textos lipsum
\usepackage{alltt}						% para colocar espaços duplos. Ex: verso livre
\usepackage{graphicx}					% para colocar imagens
\usepackage{float}						% para flutuar imagens e tabelas 		
\usepackage{lettrine}					% para capsulares
\usepackage{comment}					% para comentar o código em bloco \begin{comment}...
\usepackage{adforn}						% para adornos & glyphs
\usepackage{xcolor}					 	% para texto colorido
\usepackage[babel]{microtype}			% para ajustes finos na mancha
\usepackage{enumerate,enumitem}			% para tipos diferentes de enumeração/formatação ver `edlab-extra.sty`
\usepackage{url}					% para citar sites \url
\usepackage{marginnote}					% para notas laterais
\usepackage{titlesec}					% para produzir os distanciamentos entre pontos no \dotfill
\usepackage{textpos}
%\usepackage{makeidx} 					% para índice remissivo

\usepackage{edlab-penalties}
\usepackage{edlab-git}
\usepackage{edlab-toc}					% define sumário
\usepackage{edlab-extra}				% define epígrafe, quote
\usepackage[largepost]{edlab-margins}
\usepackage[%semcabeco, 				% para remover cabeço, sobe mancha e mantem estilos
			]{edlab-sections}			% define pagestyle (cabeço, rodapé e seções)
\usepackage[%
			% notasemlinha 			
			% notalinhalonga
			chicagofootnotes			% para notas com número e ponto cf. man. de Chicago
			]{edlab-footnotes}
\usepackage{tikz}

\babelprovide[transforms = hyphen.repeat]{portuguese}

% Medidas (ver: memoir p.11 fig.2.3)
\parindent=3ex			% Tamanho da indentação
\parskip=0pt			% Entre parágrafos
\marginparsep=1em		% Entre mancha e nota lateral 
\marginparwidth=4em		% Tamanho da caixa de texto da nota laterial

% Fontes
\newfontfamily\formular{Formular}
\newfontfamily\climate{Climate Crisis}
\newfontfamily\formularlight{Formular Light}
\setmainfont[Ligatures=TeX,Numbers=OldStyle]{Minion Pro}
\newfontfamily\arial{Arial Hebrew}        
\newfontfamily\ariallight{ArialHebrew-Light}        

% Estilos
\copypagestyle{posf}{baruch}
\makeoddhead{posf}{}{\formularlight\tiny{INTRODUÇÃO}}{}
\makeevenhead{posf}{}{\formularlight\tiny{LUIS S.\,KRAUSZ}}{}
\makeevenfoot{posf}{}{\footnotesize\thepage}{}
\makeoddfoot{posf}{}{\footnotesize\thepage}{}

\makeoddhead{baruch}{}{\formularlight\tiny\MakeUppercase{\leftmark}}{}
\makeevenhead{baruch}{}{\formularlight\tiny{VIAGEM À POLÔNIA}}{}
\makeevenfoot{baruch}{}{\footnotesize\thepage}{}
\makeoddfoot{baruch}{}{\footnotesize\thepage}{}
\pagestyle{baruch}		
\headstyles{baruch}

\begin{document}
% Este documento tem a ver com as partes do LIVRO. 

% Tamanhos
% \tiny
% \scriptsize
% \footnotesize
% \small 
% \normalsize
% \large 
% \Large 
% \LARGE 
% \huge
% \Huge

% Posicionamento
% \centering 
% \raggedright
% \raggedleft
% \vfill 
% \hfill 
% \vspace{Xcm}   % Colocar * caso esteja no começo de uma página. Ex: \vspace*{...}
% \hspace{Xcm}

% Estilo de página
% \thispagestyle{<<nosso>>}
% \thispagestyle{empty}
% \thispagestyle{plain}  (só número, sem cabeço)
% https://www.overleaf.com/learn/latex/Headers_and_footers

% Compilador que permite usar fonte de sistema: xelatex, lualatex
% Compilador que não permite usar fonte de sistema: latex, pdflatex

% Definindo fontes
% \setmainfont{Times New Roman}  % Todo o texto
% \newfontfamily\avenir{Avenir}  % Contexto

% \begin{textblock*}{2.625in}(0pt,0pt)%
% \vspace*{-2cm}
% \hspace*{-6.3cm}
% \includegraphics[scale=0.7]{./FRONTE/FRONTE.pdf}
% \end{textblock*}   

%\thispagestyle{empty}

\clearpage
\begin{tikzpicture}[remember picture, overlay]
\node at (4,-8.5) {\includegraphics[width=19cm]{./FRONTE/DOBLIN-1.jpg}}; 
\end{tikzpicture}\clearpage\thispagestyle{empty}
% \begin{tikzpicture}[remember picture,overlay]
% \fill[black] (-3,3.5) rectangle (14,-21.5);
% \end{tikzpicture}
\clearpage       % [Frontistício]
%%\newcommand{\linhalayout}[2]{{\tiny\textbf{#1}\quad#2\par}}
\newcommand{\linha}[2]{\ifdef{#2}{\linhalayout{#1}{#2}}{}}

\begingroup\tiny
\parindent=0cm
\thispagestyle{empty}

\textbf{copyright\,©}\quad			 		 {S.\,Fischer Verlag Gmb\textsc{h} 1925}\\

\textbf{edição brasileira\,©}\quad			 {Hedra \the\year}\\
\textbf{tradução do alemão e introdução\,©}\quad			 {Luis S.\,Krausz}\\
\textbf{estabelecimento das notas\,©}\quad	 {Alexandre Mazak}\\

\textbf{título original}\quad			{\textit{Reise in Polen}}\\
\textbf{imagem}\quad					{\emph{Alfred Döblin in exile}. Nova York/\,Berlim: Leo Baeck Institute}\\
\textbf{agradecimentos}\quad			{Fernando Klabin e Marion Brandt}\\

\textbf{edição}\quad			 		{Suzana Salama}\\
\textbf{editor assistente}\quad			{Paulo Henrique Pompermaier}\\
\textbf{revisão técnica}\quad			{Ernesto Mifano Honigsberg}\\
\textbf{preparação}\quad			 	{Ana Cecilia Agua de Melo}\\
\textbf{revisão}\quad			 		{Solange Mayumi Lemos}\\
\textbf{capa}\quad			 			{Lucas Kroëff}\\

\textbf{\textsc{isbn}}\quad			 	{978-65-89705-43-7}
 
\vfill

\begin{minipage}{7cm}
\textbf{Dados Internacionais de Catalogação na Publicação (\textsc{cip})}\\
Câmara Brasileira do Livro, \textsc{sp}, Brasil

\textbf{\hrule}\smallskip

\textsc{d}633v\hspace{0.5em} Döblin, Alfred, 1909--1938\\

\textit{Viagem à Polônia}. Alfred Döblin; tradução e introdução de Luis S.\,Krausz;\\ 
estabelecimento das notas de Alexandre Mazak. 1.\,ed. Título original: \textit{Reise in Polen}. São Paulo, \textsc{sp}: Hedra, 2025.\\

\textsc{isbn} 978-65-89705-43-7\\

1.\,Relato de viagem. 2.\,Polônia. 3.\,Relato de viagem. 4.\,Etnografia
europeia. 5.\,Varsóvia. 6.\,Judeus poloneses. 7.\,Judaísmo da Europa Oriental. 8.\,Assimilação cultural judaica. \textsc{i}.\,Krausz, Luis S. \textsc{ii}.\,Título. \textsc{iii}.\,Série.

\smallskip

2025--4706\hfill 			\textsc{cdd}: 910.4\\\vspace{-0.3cm}

							\hfill\textsc{cdu}: 913\,\,\,\,

\textbf{\hrule}\smallskip

\textbf{Elaborado por Eliane de Freitas Leite (\textsc{crb} 8/\,8415)}\\

\textbf{Índices para catálogo sistemático:}\\
1.\,Relato de viagem (910.4)\\
2.\,Relato de viagem (913)
\end{minipage}

\vfill

\textit{Grafia atualizada segundo o Acordo Ortográfico da Língua\\
Portuguesa de 1990, em vigor no Brasil desde 2009.}\\

\textit{Direitos reservados em língua\\ 
portuguesa somente para o Brasil.}\\

\textsc{editora hedra ltda.}\\
R.\,Sete de Abril, 235, cj.\,102\\
01043--904 São Paulo \textsc{sp} Brasil\\
Telefone +55 11 3097 8304\\\smallskip
editora@hedra.com.br\\
www.hedra.com.br

\endgroup
\pagebreak     % [Créditos]
% Tamanhos
% \tiny
% \scriptsize
% \footnotesize
% \small 
% \normalsize
% \large 
% \Large 
% \LARGE 
% \huge
% \Huge

% Posicionamento
% \centering 
% \raggedright
% \raggedleft
% \vfill 
% \hfill 
% \vspace{Xcm}   % Colocar * caso esteja no começo de uma página. Ex: \vspace*{...}
% \hspace{Xcm}

% Estilo de página
% \thispagestyle{<<nosso>>}
% \thispagestyle{empty}
% \thispagestyle{plain}  (só número, sem cabeço)
% https://www.overleaf.com/learn/latex/Headers_and_footers

% Compilador que permite usar fonte de sistema: xelatex, lualatex
% Compilador que não permite usar fonte de sistema: latex, pdflatex

% Definindo fontes
% \setmainfont{Times New Roman}  % Todo o texto
% \newfontfamily\avenir{Avenir}  % Contexto

\begingroup\thispagestyle{empty}\vspace*{.05\textheight} 

              \formular
              
              \LARGE\noindent
              Receitas

              % \smallskip
                      
              % \large\noindent\textit{Diário de sanatório}

              % \bigskip  
              
              % \Large\noindent\textbf{Alfred Döblin}
              
              %\vspace{3.4em}

              % \vfill
              % %\newfontfamily\minion{Minion Pro}
              % {\fontsize{30}{40}%\selectfont\minion
              % \small\noindent 
              % Luis S.\,Krausz (\textit{tradução e introdução})\vspace{-0.3em}\\
              % \noindent
              % Alexandre Mazak (\textit{estabelecimento das notas})
              % }
              
              % %\vspace{3.5em}
              % \bigskip

              % \noindent
              % {\fontsize{30}{40}%\selectfont\minion
              % \small\noindent 1ª edição}
                      
              % \vfill
              % %\noindent\includegraphics[width=0.21\textwidth]{logo}
              % %\break{} 

              % {\newfontfamily\timesnewroman{Times New Roman}
              % {\noindent\fontsize{30}{40}\selectfont\timesnewroman hedra}}

              % \vspace{-0.1cm}
              % {\fontsize{30}{40}%\selectfont\minion
              % \scriptsize\noindent São Paulo\quad\the\year}

\endgroup
\pagebreak
	       % [folha de rosto]
% nothing			is level -3
% \book				is level -2
% \part				is level -1
% \chapter 			is level 0
% \section 			is level 1
% \subsection 		is level 2
% \subsubsection 	is level 3
% \paragraph 		is level 4
% \subparagraph 	is level 5
\setcounter{secnumdepth}{-2}
\setcounter{tocdepth}{0}

% \renewcommand{\contentsname}{Índex} 	% Trocar nome do sumário para 'Índex'
%\ifodd\thepage\relax\else\blankpage\fi 	% Verifica se página é par e coloca página branca
%\tableofcontents*

\pagebreak
\begingroup \footnotesize \parindent0pt \parskip 5pt \thispagestyle{empty} \vspace*{-0.5\textheight}\mbox{} \vfill
\baselineskip=.92\baselineskip
\textbf{Receitas} \lipsum[2]

\textbf{Eliane Salama} \lipsum[2]


\endgroup

\pagebreak
{\begingroup\mbox{}\pagestyle{empty}
\pagestyle{empty} 
% \renewcommand{\contentsname}{Índex} 	% Trocar nome do sumário para 'Índex'
%\ifodd\thepage\relax\else\blankpage\fi 	% Verifica se página é par e coloca página branca
\addtocontents{toc}{\protect\thispagestyle{empty}}
\tableofcontents*\clearpage\endgroup}

\part{Receitas}

\chapter{Tarte noix de coco}

1 verre noix de coco
5 œufs, 2 verres farine
2 verres sucre, 1 verre lait
1 paquet beurre, 1 c. à s.
baking, vanille

Travailler le beurre avec
1 verre sucre, ajouter les
œufs en travaillant
ainsi que la farine, lait,
baking et vanille.

Prendre ensuite 1 verre sucre
de l’eau et citron, faire
bouillir et retirer, mouler le
gâteau avec 1 fourchette
imbiber le gâteau avec
le jus et un peu de liqueur.

%\blankpage

\pagebreak

%\blankAteven
\pagestyle{empty}
\begingroup
\rmfamily
\fontsize{7}{8}\selectfont

\color{black}


% {%
%   \IfFileExists{img/HEDRA_EDICOES.pdf}{%
%     \vspace*{0mm}%
%     \begin{enumerate}
%       \setlength{\topsep}{0pt}\setlength{\partopsep}{0pt}\setlength{\itemsep}{0pt}\setlength{\parsep}{0pt}
%       \item[] \noindent\hspace*{-1em}\includegraphics[width=.42\textwidth,keepaspectratio]{img/HEDRA_EDICOES.pdf}\par
%     \end{enumerate}
%   }{\large\textsc{hedra edições}}%
% }



% % Hide enumerate labels but keep default indentation
% \makeatletter
% \renewcommand{\labelenumi}{}
% \renewcommand{\labelenumii}{}
% \renewcommand{\labelenumiii}{}
% \makeatother

% \begin{enumerate}
% \setlength{\topsep}{2pt}
% \setlength{\partopsep}{0pt}
% \setlength\parskip{4.2pt}
% \setlength\itemsep{-1.4mm}
% \item \textit{A arte da guerra}, Maquiavel
% \item \textit{A cruzada das crianças\,/\,Vidas imaginárias}, Marcel Schwob
% \item \textit{A filosofia na era trágica dos gregos}, Friedrich Nietzsche
% \item \textit{A fábrica de robôs}, Karel Tchápek 
% \item \textit{A história trágica do Doutor Fausto}, Christopher Marlowe
% \item \textit{A metamorfose}, Franz Kafka
% \item \textit{A monadologia e outros textos}, Gottfried Leibniz
% \item \textit{A morte de Ivan Ilitch}, Lev Tolstói 
% \item \textit{A velha Izerguil e outros contos}, Maksim Górki
% \item \textit{A vida é sonho}, Calderón de la Barca
% \item \textit{A volta do parafuso}, Henry James
% \item \textit{A voz dos botequins e outros poemas}, Paul Verlaine 
% \item \textit{A vênus das peles}, Leopold von Sacher{}-Masoch
% \item \textit{A última folha e outros contos}, O.\,Henry
% \item \textit{Americanismo e fordismo}, Antonio Gramsci
% \item \textit{Apologia de Galileu}, Tommaso Campanella 
% \item \textit{Arcana C\oe lestia} e \textit{Apocalipsis revelata}, Emanuel Swedenborg
% \item \textit{Autobiografia de uma pulga}, [Stanislas de Rhodes]
% \item \textit{Balada dos enforcados e outros poemas}, François Villon
% \item \textit{Carmilla, a vampira de Karnstein}, Sheridan Le Fanu
% \item \textit{Carta sobre a tolerância}, John Locke
% \item \textit{Contos clássicos de vampiro}, L.\,Byron, B.\,Stoker \& outros
% \item \textit{Contos de amor, de loucura e de morte}, Horacio Quiroga
% \item \textit{Contos indianos}, Stéphane Mallarmé
% \item \textit{Cultura estética e liberdade}, Friedrich von Schiller
% \item \textit{Dao De Jing}, Lao Zi
% \item \textit{Discursos ímpios}, Marquês de Sade
% \item \textit{Dissertação sobre as paixões}, David Hume
% \item \textit{Diário de um escritor} (1873), Fiódor Dostoiévski
% \item \textit{Diário parisiense e outros escritos}, Walter Benjamin
% \item \textit{Diários de Adão e Eva}, Mark Twain
% \item \textit{Don Juan}, Molière
% \item \textit{Dos novos sistemas na arte}, Kazimir Maliévitch
% \item \textit{Educação e sociologia}, Émile Durkheim
% \item \textit{Elogio da loucura}, Erasmo de Rotterdam
% \item \textit{Emília Galotti}, Gotthold Ephraim Lessing
% \item \textit{Ernestine ou o nascimento do amor}, Stendhal
% \item \textit{Escritos sobre arte}, Charles Baudelaire
% \item \textit{Escritos sobre literatura}, Sigmund Freud
% \item \textit{Eu acuso!}, Zola\,/\,\textit{O processo do capitão Dreyfus}, Rui Barbosa
% \item \textit{Explosão: romance da etnologia}, Hubert Fichte
% \item \textit{Feitiço de amor e outros contos}, Ludwig Tieck
% \item \textit{Flossie, a Vênus de quinze anos}, [Swinburne]
% \item \textit{Fábula de Polifemo e Galateia e outros poemas}, Góngora
% \item \textit{Fé e saber}, Georg W.\,F.\,Hegel
% \item \textit{Gente de Hemsö}, August Strindberg 
% \item \textit{Hawthorne e seus musgos}, Herman Melville
% \item \textit{Imitação de Cristo}, Tomás de Kempis
% \item \textit{Incidentes da vida de uma escrava}, Harriet Jacobs
% \item \textit{Inferno}, August Strindberg
% \item \textit{Investigação sobre o entendimento humano}, David Hume
% \item \textit{Jazz rural}, Mário de Andrade
% \item \textit{Jerusalém}, William Blake
% \item \textit{Joana d'Arc}, Jules Michelet
% \item \textit{Ludwig Feuerbach e o fim da filosofia clássica alemã}, Friedrich Engels
% \item \textit{Manifesto comunista}, Karl Marx e Friedrich Engels
% \item \textit{Memórias do subsolo}, Fiódor Dostoiévski
% \item \textit{Micromegas e outros contos}, Voltaire
% \item \textit{Narrativa de William W.\,Brown, escravo fugitivo}, William Wells Brown
% \item \textit{Nascidos na escravidão: depoimentos norte-americanos}, \textsc{wpa}
% \item \textit{No coração das trevas}, Joseph Conrad
% \item \textit{Noites egípcias e outros contos}, Aleksandr Púchkin
% \item \textit{O casamento do Céu e do Inferno}, William Blake
% \item \textit{O cego e outros contos}, \textsc{d.\,h}.\,Lawrence
% \item \textit{O chamado de Cthulhu}, \textsc{h.\,p.}\,Lovecraft
% \item \textit{O contador de histórias e outros textos}, Walter Benjamin
% \item \textit{O corno de si próprio e outros contos}, Marquês de Sade
% \item \textit{O destino do erudito}, Johann Fichte
% \item \textit{O estranho caso do dr.\,Jekyll e Mr. Hyde}, Robert Louis Stevenson
% \item \textit{O fim do ciúme e outros contos}, Marcel Proust
% \item \textit{O ladrão honesto e outros contos}, Fiódor Dostoiévski
% \item \textit{O livro de Monelle}, Marcel Schwob
% \item \textit{O mundo ou tratado da luz}, René Descartes
% \item \textit{O novo Epicuro: as delícias do sexo}, Edward Sellon
% \item \textit{O pequeno Zacarias, chamado Cinábrio}, \textsc{e.\,t.\,a.},\,Hoffmann
% \item \textit{O primeiro Hamlet}, William Shakespeare
% \item \textit{O príncipe}, Maquiavel
% \item \textit{O que eu vi, o que nós veremos}, Santos-Dumont
% \item \textit{O retrato de Dorian Gray}, Oscar Wilde
% \item \textit{O sobrinho de Rameau}, Diderot
% \item \textit{Ode ao Vento Oeste e outros poemas}, \textsc{p.\,b.},\,Shelley
% \item \textit{Ode sobre a melancolia e outros poemas}, John Keats
% \item \textit{Oliver Twist}, Charles Dickens
% \item \textit{Os sofrimentos do jovem Werther}, Goethe
% \item \textit{Para serem lidas à noite}, Ion Minulescu
% \item \textit{Pensamento político de Maquiavel}, Johann Fichte
% \item \textit{Pequeno-burgueses}, Maksim Górki
% \item \textit{Pequenos poemas em prosa}, Charles Baudelaire
% \item \textit{Perversão: a forma erótica do ódio}, Robert Stoller
% \item \textit{Poemas}, Lord Byron
% \item \textit{Poesia basca: das origens à Guerra Civil} 
% \item \textit{Poesia catalã: das origens à Guerra Civil} 
% \item \textit{Poesia espanhola: das origens à Guerra Civil} 
% \item \textit{Poesia galega: das origens à Guerra Civil} 
% \item \textit{Poesia em tempo difíceis}, Berlolt Brecht 
% \item \textit{Pr\ae terita}, John Ruskin
% \item \textit{Rashômon e outros contos}, Ryūnosuke Akutagawa
% \item \textit{Robinson Crusoé}, Daniel Defoe
% \item \textit{Romanceiro cigano}, Federico García Lorca
% \item \textit{Sagas}, August Strindberg
% \item \textit{Sobre a amizade e outros diálogos}, Jorge Luis Borges e Osvaldo Ferrari
% \item \textit{Sobre a filosofia e outros diálogos}, Jorge Luis Borges e Osvaldo Ferrari
% \item \textit{Sobre a filosofia e seu método (Parerga e paralipomena)}, Arthur Schopenhauer 
% \item \textit{Sobre a liberdade}, Stuart Mill
% \item \textit{Sobre a utilidade e a desvantagem da histório para a vida}, Friedrich Nietzsche
% \item \textit{Sobre a ética (Parerga e paralipomena)}, Arthur Schopenhauer 
% \item \textit{Sobre os sonhos e outros diálogos}, Jorge Luis Borges e Osvaldo Ferrari
% \item \textit{Sobre verdade e mentira}, Friedrich Nietzsche
% \item \textit{Sonetos}, William Shakespeare
% \item \textit{Sátiras, fábulas, aforismos e profecias}, Leonardo da Vinci
% \item \textit{Teleny, ou o reverso da medalha}, Oscar Wilde
% \item \textit{Triunfos}, Petrarca
% \item \textit{Um anarquista e outros contos}, Joseph Conrad
% \item \textit{Viagem aos Estados Unidos}, Alexis de Tocqueville
% \item \textit{Viagem em volta do meu quarto}, Xavier de Maistre 
% \item \textit{Viagem sentimental}, Laurence Sterne
% \item \textit{Émile e Sophie ou os solitários}, Jean-Jacques Rousseau 
% \end{enumerate}



% \IfFileExists{img/METABIBLIOTECA.pdf}{%
%   \vspace*{0mm}%
%   \begin{enumerate}
%     \setlength{\topsep}{0pt}\setlength{\partopsep}{0pt}\setlength{\itemsep}{0pt}\setlength{\parsep}{0pt}
%     \item[] \noindent\hspace*{-1em}\includegraphics[width=.42\textwidth,keepaspectratio]{img/METABIBLIOTECA.pdf}\par
%   \end{enumerate}
% }{\large\textsc{metabiblioteca}}



% \begin{enumerate}
% \setlength{\topsep}{2pt}
% \setlength{\partopsep}{0pt}
% \setlength\parskip{4.2pt}
% \setlength\itemsep{-1.4mm}
% \item \textit{A carteira de meu tio}, Joaquim Manuel de Macedo
% \item \textit{A cidade e as serras}, Eça de Queirós
% \item \textit{A escrava}, Maria Firmina dos Reis
% \item \textit{A família Medeiros}, Júlia Lopes de Almeida 
% \item \textit{A pele do lobo e outras peças}, Artur Azevedo
% \item \textit{Auto da barca do inferno}, Gil Vicente
% \item \textit{Bom crioulo}, Adolfo Caminha
% \item \textit{Cartas a favor da escravidão}, José de Alencar
% \item \textit{Contos e novelas}, Júlia Lopes de Almeida
% \item \textit{Crime}, Luiz Gama
% \item \textit{Democracia}, Luiz Gama
% \item \textit{Direito}, Luiz Gama
% \item \textit{Elixir do pajé: poemas de humor, sátira e escatologia}, Bernardo Guimarães
% \item \textit{Eu}, Augusto dos Anjos
% \item \textit{Farsa de Inês Pereira}, Gil Vicente
% \item \textit{Helianto}, Orides Fontela
% \item \textit{História da província Santa Cruz}, Gandavo
% \item \textit{Índice das coisas mais notáveis}, Antônio Vieira
% \item \textit{Iracema}, José de Alencar
% \item \textit{Liberdade}, Luiz Gama
% \item \textit{Mensagem}, Fernando Pessoa
% \item \textit{Meridiano 55}, Flávio de Carvalho
% \item \textit{O Ateneu}, Raul Pompeia
% \item \textit{O cortiço}, Aluísio Azevedo
% \item \textit{O desertor}, Silva Alvarenga
% \item \textit{Oração aos moços}, Rui Barbosa
% \item \textit{Pai contra mãe: as origens perdidas de Luiz Gama}, Bruno Rodrigues de Lima
% \item \textit{Pai contra mãe e outros contos}, Machado de Assis
% \item \textit{Poemas completos de Alberto Caeiro}, Fernando Pessoa
% \item \textit{Teatro de êxtase}, Fernando Pessoa
% \item \textit{Transposição}, Orides Fontela
% \item \textit{Tratado descritivo do Brasil em 1587}, Gabriel Soares de Sousa
% \item \textit{Tratados da terra e gente do Brasil}, Fernão Cardim 
% \item \textit{Utopia Brasil}, Darcy Ribeiro
% \end{enumerate}



\IfFileExists{img/AYLLON.pdf}{%
  \vspace*{0mm}%
  \begin{enumerate}
    \setlength{\topsep}{0pt}\setlength{\partopsep}{0pt}\setlength{\itemsep}{0pt}\setlength{\parsep}{0pt}
    \item[] \noindent\hspace*{-1em}\includegraphics[width=.42\textwidth,keepaspectratio]{img/AYLLON.pdf}\par
  \end{enumerate}
}{\large\textsc{ayllon}}



\begin{enumerate}
\setlength{\topsep}{2pt}
\setlength{\partopsep}{0pt}
\setlength\parskip{4.2pt}
\setlength\itemsep{-1.4mm}
\item \textit{A toca iluminada}, Max Blecher
\item \textit{Acontecimentos na irrealidade imediata}, Max Blecher
\item \textit{Cabalat shabat: poemas rituais}, Fabiana Gampel Grinberg
\item \textit{Em busca de meus irmãos na América}, Chaim Novodvorsky
\item \textit{Fragmentos de um diário encontrado}, Mihail Sebastian
\item \textit{Israel e Palestina}, Gershon Baskin
\item \textit{Mulheres}, Mihail Sebastian
\item \textit{O Rabi de Bacherach}, Heinrich Heine
\item \textit{Viagem à Polônia}, Alfred Döblin
\item \textit{Vilna: cidade dos outros}, Laimonas Briedis
\end{enumerate}



\IfFileExists{img/PAIDEIA.pdf}{%
  \vspace*{0mm}%
  \begin{enumerate}
    \setlength{\topsep}{0pt}\setlength{\partopsep}{0pt}\setlength{\itemsep}{0pt}\setlength{\parsep}{0pt}
    \item[] \noindent\hspace*{-1em}\includegraphics[width=.42\textwidth,keepaspectratio]{img/PAIDEIA.pdf}\par
  \end{enumerate}
}{\large\textsc{paideia}}



\begin{enumerate}
\setlength{\topsep}{2pt}
\setlength{\partopsep}{0pt}
\setlength\parskip{4.2pt}
\setlength\itemsep{-1.4mm}
\item \textit{A conjuração de Catilina}, Salústio
\item \textit{As bacantes}, Eurípides
\item \textit{Cântico dos cânticos}, [Salomão]
\item \textit{Édipo Rei}, Sófocles
\item \textit{Fedro}, Platão
\item \textit{Hino a Afrodite e outros poemas}, Safo de Lesbos 
\item \textit{Lira grega}, Giuliana Ragusa (org.)
\item \textit{Lisístrata}, Aristófanes
\item \textit{Metamorfoses}, Ovídio
\item \textit{Primeiro livro dos Amores}, Ovídio
\item \textit{Sobre o riso e a loucura}, [Hipócrates]
\item \textit{Teogonia}, Hesíodo
\item \textit{Trabalhos e dias}, Hesíodo
\end{enumerate}



% \IfFileExists{img/QUE_HORAS_SAO.pdf}{%
%   \vspace*{0mm}%
%   \begin{enumerate}
%     \setlength{\topsep}{0pt}\setlength{\partopsep}{0pt}\setlength{\itemsep}{0pt}\setlength{\parsep}{0pt}
%     \item[] \noindent\hspace*{-1em}\includegraphics[width=.42\textwidth,keepaspectratio]{img/QUE_HORAS_SAO.pdf}\par
%   \end{enumerate}
% }{\large\textsc{que horas são?}}



% \begin{enumerate}
% \setlength{\topsep}{2pt}
% \setlength{\partopsep}{0pt}
% \setlength\parskip{4.2pt}
% \setlength\itemsep{-1.4mm}
% \item \textit{8/1: A rebelião dos manés}, Pedro Fiori Arantes, Fernando Frias e Maria Luiza Meneses
% \item \textit{A linguagem fascista}, Carlos Piovezani \& Emilio Gentile
% \item \textit{A sociedade de controle}, J.\,Souza; R.\,Avelino; S.\,Amadeu (org.)
% \item \textit{As Big Techs e a guerra total: o complexo militar-industrial-dataficado}, Sérgio A. da Silveira
% \item \textit{Ativismo digital hoje}, R.\,Segurado; C.\,Penteado; S.\,Amadeu (org.)
% \item \textit{Crédito à morte}, Anselm Jappe
% \item \textit{Descobrindo o Islã no Brasil}, Karla Lima
% \item \textit{Desinformação e democracia}, Rosemary Segurado
% \item \textit{Dilma Rousseff e o ódio político}, Tales Ab'Sáber
% \item \textit{Labirintos do fascismo} (v.\,\textsc{i}), João Bernardo
% \item \textit{Labirintos do fascismo} (v.\,\textsc{ii}), João Bernardo
% \item \textit{Labirintos do fascismo} (v.\,\textsc{iii}), João Bernardo
% \item \textit{Labirintos do fascismo} (v.\,\textsc{iv}), João Bernardo
% \item \textit{Labirintos do fascismo} (v.\,\textsc{v}), João Bernardo
% \item \textit{Labirintos do fascismo} (v.\,\textsc{vi}), João Bernardo
% \item \textit{Lugar de negro, lugar de branco?}, Douglas Rodrigues Barros
% \item \textit{Lulismo, carisma pop e cultura anticrítica}, Tales Ab'Sáber
% \item \textit{Machismo, racismo, capitalismo identitário}, Pablo Polese
% \item \textit{Michel Temer e o fascismo comum}, Tales Ab'Sáber
% \item \textit{O quarto poder: uma outra história}, Paulo Henrique Amorim
% \item \textit{Universidade, cidade e cidadania}, Franklin Leopoldo e Silva
% \end{enumerate}



\IfFileExists{img/MUNDO_INDIGENA.pdf}{%
  \vspace*{0mm}%
  \begin{enumerate}
    \setlength{\topsep}{0pt}\setlength{\partopsep}{0pt}\setlength{\itemsep}{0pt}\setlength{\parsep}{0pt}
    \item[] \noindent\hspace*{-1em}\includegraphics[width=.42\textwidth,keepaspectratio]{img/MUNDO_INDIGENA.pdf}\par
  \end{enumerate}
}{\large\textsc{mundo indígena}}



\begin{enumerate}
\setlength{\topsep}{2pt}
\setlength{\partopsep}{0pt}
\setlength\parskip{4.2pt}
\setlength\itemsep{-1.4mm}
\item \textit{A árvore dos cantos}, Pajés Parahiteri
\item \textit{A folha divina}, Timóteo Verá Tupã Popygua
\item \textit{A mulher que virou tatu}, Eliane Camargo
\item \textit{A terra uma só}, Timóteo Verá Tupã Popygua
\item \textit{Cantos dos animais primordiais}, Ava Ñomoandyja Atanásio Teixeira
\item \textit{Círculos de coca e fumaça}, Danilo Paiva Ramos
\item \textit{Crônicas de caça e criação}, Uirá Garcia
\item \textit{Nas redes guarani}, Valéria Macedo \& Dominique Tilkin-Gallois
\item \textit{Não havia mais homens}, Luciana Storto
\item \textit{O surgimento da noite}, Pajés Parahiteri
\item \textit{O surgimento dos pássaros}, Pajés Parahiteri
\item \textit{Os Aruaques}, Max Schmidt
\item \textit{Os cantos do homem-sombra}, Patience Epps e Danilo Paiva Ramos
\item \textit{Os comedores de terra}, Pajés Parahiteri
\item \textit{Xamanismos ameríndios}, A.\,Barcelos Neto, L.\,Pérez Gil \& D.\,Paiva Ramos
\end{enumerate}



% % \IfFileExists{img/narrativas.eps}{%
% %  \vspace*{0mm}\noindent\includegraphics[width=.42\textwidth,keepaspectratio]{img/narrativas.eps}\par
% %}{\large\textsc{narrativas da escravidão}}

% %

% %\begin{enumerate}
% %\setlength{\topsep}{2pt}
% %\setlength{\partopsep}{0pt}
% %\setlength\parskip{4.2pt}
% %\setlength\itemsep{-1.4mm}
% %\item \textit{Incidentes da vida de uma escrava}, Harriet Jacobs
% %\item \textit{Nascidos na escravidão: depoimentos norte-americanos}, \textsc{wpa}
% %\item \textit{Narrativa de William W. Brown, escravo fugitivo}, William Wells Brown
% %\end{enumerate}

% %

% \IfFileExists{img/ANARC.pdf}{%
%   \vspace*{0mm}%
%   \begin{enumerate}
%     \setlength{\topsep}{0pt}\setlength{\partopsep}{0pt}\setlength{\itemsep}{0pt}\setlength{\parsep}{0pt}
%     \item[] \noindent\hspace*{-1em}\includegraphics[width=.42\textwidth,keepaspectratio]{img/ANARC.pdf}\par
%   \end{enumerate}
% }{\large\textsc{anarc}}



% \begin{enumerate}
% \setlength{\topsep}{2pt}
% \setlength{\partopsep}{0pt}
% \setlength\parskip{4.2pt}
% \setlength\itemsep{-1.4mm}
% \item \textit{Ação direta}, Voltairine de Cleyre
% \item \textit{Anarquia pela educação}, Élisée Reclus
% \item \textit{Entre camponeses}, Malatesta
% \item \textit{Escritos revolucionários}, Malatesta
% \item \textit{História da anarquia} (v.\,1), Max Nettlau
% \item \textit{História da anarquia} (v.\,2), Max Nettlau
% \item \textit{O indivíduo, a sociedade e o Estado, e outros ensaios}, Emma Goldman
% \item \textit{O princípio anarquista e outros ensaios}, Kropotkin
% \item \textit{O princípio do Estado e outros ensaios}, Bakunin
% \item \textit{Os sovietes traídos pelos bolcheviques}, Rocker
% \item \textit{Revolução e liberdade: cartas de 1845 a 1875}, Bakunin
% \item \textit{Sobre anarquismo, sexo e casamento}, Emma Goldman
% \end{enumerate}


%

%\IfFileExists{img/ecopolitica.eps}{%
%  \vspace*{0mm}\noindent\includegraphics[width=.42\textwidth,keepaspectratio]{img/ecopolitica.eps}\par
%}{\large\textsc{ecopolítica}}

%

%\begin{enumerate}
%\setlength{\topsep}{2pt}
%\setlength{\partopsep}{0pt}
%\setlength\parskip{4.2pt}
%\setlength\itemsep{-1.4mm}
%\item \textit{Anarquistas na América do Sul}, E.\,Passetti, S.\,Gallo; A.\,Augusto  (org.)
%\item \textit{Ecopolítica}, E.\,Passetti; A.\,Augusto; B.\,Carneiro; S.\,Oliveira, T.\,Rodrigues  (org.)
%\item \textit{Pandemia e anarquia}, E.\,Passetti; J.\,da Mata; J.\,Ferreira  (org.)
%\end{enumerate}

\endgroup % fecha o grupo
\nopagecolor
\color{black}
\pagebreak	   % [lista de livros publicados]
%\blankpage

\ifodd\thepage\blankpage\fi

\parindent=0pt
\footnotesize\thispagestyle{empty}

\newfontfamily\minion{Minion Pro}

\mbox{}\vfill
\begin{center}
		\begin{minipage}{.7\textwidth}\tiny\noindent{}
		\centering\tiny\minion
		Adverte-se aos curiosos que se imprimiu este 
		livro na gráfica Expressão e Arte, em papel Pólen Bold 70, composto em tipologia Minion Pro, em 11 pontos, com diversos sofwares livres, 
		entre eles, {Lua\LaTeX} e git.\\ 
		\ifdef{\RevisionInfo{}}{\par(v.\,\RevisionInfo)}{}\medskip\\\
		\adforn{64}
		\end{minipage}
\end{center}		   % [colofon]
\checkandfixthelayout
\end{document}
